\documentclass[class=report, crop=false]{standalone}
\input{preamble}

\begin{document}
  \chapter{Sets}

  \section{Introduction to Sets}

  No exercises.

  \section{Properties}

  No exercises

  \section{The Axioms}

  % Problem 3.1
  \begin{problem}
    Show that the set of all \(x\) such that \(x \in A\) and \(x \notin B\) exists.
  \end{problem}

  \begin{solution}
    Consider the property \(\textbf{P}(x, B)\): ``\(x \notin B\)." Then, by the Comprehension Schema, for every \(A\) and \(B\) there exists a set \(C\) such that \(x \in C\) if and only if \(x \in A\) and \(\textbf{P}(x, B)\), or if and only if \(x \in A\) and \(x \notin B\).
  \end{solution}


  % Problem 3.2
  \begin{problem}
    Replace the Axiom of Existence by the following weaker postulate:
    \begin{axiom}[Weak Axiom of Existence]
      Some set exists.
    \end{axiom}
    Prove the Axiom of Existence using the Weak Axiom of Existence and the Comprehension Schema. [\textit{Hint}: Let \(A\) be a set known to exist; consider \(\{x \in A \mid x \neq x\}\).]
  \end{problem}

  \begin{solution}
    Recall that the Axiom of Existence states that there exists a set which has no elements. Let \(A\) be a set known to exist, which is guaranteed by the Weak Axiom of Existence. Consider the property \(\textbf{P}(x)\): ``\(x \neq x\)."
    Then, by the Comprehension Schema, there is a set \(B\) such that \(x \in B\) if and only if \(x \in A\) and \(\textbf{P}(x)\), or \(x \in A\) and \(x \neq x\). However, we have that \(\forall x : x = x\). That is, there is no \(x\) such that \(x \neq x\). In particular, the set \(B\) has no elements so there exists a set which has no elements.
  \end{solution}


  % Problem 3.3
  \begin{problem}
    \hfill
    \begin{enumerate}[label={(\alph*)}]
      \item Prove that a "set of all sets" does not exist. [\textit{Hint}: if \(V\) is a set of all sets, consider \(\{x \in V \mid x \notin x\)\}.]

      \item Prove that for any set \(A\) there is some \(x \notin A\).
    \end{enumerate}
  \end{problem}

  \begin{solution}
    \hfill
    \begin{enumerate}[label={(\alph*)}]
      \item Suppose that there exists a set \(V\) containing all sets. Consider the property \(\textbf{P}(x)\): ``\(x \notin x\)." By the Comprehension Schema, there exists a set \(X = \{x \in V \mid \textbf{P}(x)\} = \{x \in V \mid x \notin x\}\). That is, \(X \in V\) because it's a set.

      We either have that \(X \in X\) or \(X \notin X\). If \(X \in X\), then \(X \in V\) and \(X \notin X\), a contradiction. If \(X \notin X\), then \(X \in V\) and \(X \notin X\) imply that \(X \in X\), another contradiction.
      Since every step is true, it must be the case that our original assumption of the existence of \(V\) is false.

      \item Suppose there is a set \(A\) such that \(\forall x : x \in A\). Since every object \(x\) we have constructed so far is a set, \(A\) is a "set of all sets" which we have shown cannot exist.
    \end{enumerate}
  \end{solution}


  % Problem 3.4
  \begin{problem}
    Let \(A\) and \(B\) be sets. Show that there exists a unique set \(C\) such that \(x \in C\) if and only if either \(x \in A\) and \(x \notin B\) or \(x \in B\) and \(x \notin A\).
  \end{problem}

  \begin{solution}
    By Problem 1.3.1, the following two sets exist:
    \begin{gather*}
      C_{1} = \{x \mid x \in A \text{ and } x \notin B\} \\
      C_{2} = \{x \mid x \in B \text{ and } x \notin A\}
    \end{gather*}
    By the Axiom of Union, there is a set \(C = C_{1} \cup C_{2}\) such that \(x \in C\) if and only if \(x \in C_{1}\) or \(x \in C_{2}\). That is,
    \begin{gather*}
      C = \{x \mid \left(x \in A \text{ and } x \notin B\right) \text{ or } \left(x \in B \text{ and } x \notin A\right)\}
    \end{gather*}
    Thus, \(C\) exists. If \(C'\) is another set satisfying the hypothesis, then \(x \in C\) if and only if \(x \in C'\) so by the Axiom of Extensionality, \(C = C'\). That is, \(C\) is unique. In general, the union of sets is unique.
  \end{solution}

  \pagebreak

  % Problem 3.5
  \begin{problem}
    \hfill
    \begin{enumerate}[label={(\alph*)}]
      \item Given \(A, B, \text{ and } C\), there is a set \(P\) such that \(x \in P\) if and only if \(x = A\) or \(x = B\) or \(x = C\).

      \item Generalize to four elements.
    \end{enumerate}
  \end{problem}

  \begin{solution}
    \hfill
    \begin{enumerate}[label={(\alph*)}]
      \item By the Axiom of Pair, the following three sets exist:
      \begin{gather*}
        P_{A} = \{x \mid x = A\} = \{A\} \\
        P_{B} = \{x \mid x = B\} = \{B\} \\
        P_{C} = \{x \mid x = C\} = \{C\}
      \end{gather*}
      Then by the Axiom of Union, there is a set \(P = \bigcup \{P_{A}, P_{B}, P_{C}\}\) such that \(x \in P\) if and only if \(x \in S\) for some \(S \in P\).
      That is, \(x \in S\) if and only if \(x \in P_{A}\) or \(x \in P_{B}\) or \(x \in P_{C}\) if and only if \(x = A\) or \(x = B\) or \(x = C\).

      \item Again by the Axiom of Pair, there exists a set \(P_{D} = \{x \mid x = D\} = \{D\}\). Again by the Axiom of Union, there is a set \(P = \bigcup \{P_{A}, P_{B}, P_{C}, P_{D}\}\) such that \(x \in P\) if and only if \(x = A\) or \(x = B\) or \(x = C\) or \(x = D\).
    \end{enumerate}
  \end{solution}


  % Problem 3.6
  \begin{problem}
    Show that \(\mathcal{P}(X) \subseteq X\) is false for any \(X\). In particular, \(\mathcal{P}(X) \neq X\) for any \(X\). This proves again that a "set of all sets" does not exist.
    [\textit{Hint}: Let \(Y = \{u \in X \mid u \notin u\}\); \(Y \in \mathcal{P}(X)\) but \(Y \notin X\).]
  \end{problem}

  \begin{solution}
    Let \(X\) be any set and define \(Y = \{u \in X \mid u \notin u\}\). Certainly \(Y \subseteq X\) because \(x \in Y\) implies \(x \in X\). Thus, \(Y \in \mathcal{P}(X)\) by the Axiom of Powerset.
    Suppose \(Y \in X\). We either have \(Y \in Y\) or \(Y \notin Y\). If \(Y \in Y\) then we have \(Y \notin Y\), a contradiction. On the other hand, if \(Y \notin Y\), then it is implied that \(Y \in Y\), another contradiction.
    Thus, our assumption that \(Y \in X\) must be false. Therefore, there is an element in \(\mathcal{P}(X)\) which does not belong to \(X\) so \(\mathcal{P}(X) \not\subseteq X\).
  \end{solution}


  % Problem 3.7
  \begin{problem}
    The Axiom of Pair, the Axiom of Union, and the Axiom of Power Set can be replaced by the following weaker versions.
    \begin{axiom}[Weak Axiom of Pair]
      For any \(A\) and \(B\), there is a set \(C\) such that \(A \in C\) and \(B \in C\).
    \end{axiom}
    \begin{axiom}[Weak Axiom of Union]
      For any set \(S\), there exists \(U\) such that if \(X \in A\) and \(A \in S\), then \(X \in U\).
    \end{axiom}
    \begin{axiom}[Weak Axiom of Power Set]
      For any set \(S\), there exists \(P\) such that \(X \subseteq S\) implies \(X \in P\).
    \end{axiom}
    Prove the Axiom of Pair, the Axiom of Union, and the Axiom of Power Set using these weaker versions. [\textit{Hint}: Use also the Comprehension Schema.]
  \end{problem}

  \begin{solution}
    By the Weak Axiom of Pair, there exists a set \(C'\) such that \(A \in C'\) and \(B \in C'\). Consider the property \(\textbf{P}(x, A, B)\): ``\(x = A\) or \(x = B\)." By the Comprehension Schema, there exists a set \(C\) such that \(x \in C\) if and only if \(x \in C'\) and \(\textbf{P}(x, A, B)\).
    That is, for any sets \(A\) and \(B\), there exists a set \(C\) such that \(x \in C\) if and only if \(x = A\) or \(x = B\), which is the Axiom of Pair.

    By the Weak Axiom of Union, there exists a set \(U'\) for all \(S\) such that if \(X \in A\) and \(A \in S\) then \(X \in U'\). Consider the property \(\textbf{P}(X, A, S)\): ``\(\exists A \in S\) such that \(X \in A\)." By the Comprehension Schema, there exists a set \(U\) such that \(X \in U\) if and only if \(X \in U'\) and \(\textbf{P}(X, A, S)\).
    That is, for any set \(S\), there exists a set \(U\) such that \(X \in U\) if and only if there exists \(A \in S\) such that \(X \in A\), which is the Axiom of Union.

    By the Axiom of Power Set, there exists a set \(P'\) such that \(X \in P'\) if and only if \(X \subseteq S\). Consider the property \(\textbf{P}(X, S)\): ``\(X \subseteq S\)." By the Comprehension Schema, there exists a set \(P\) such that \(X \in C\) if and only if \(X \in P'\) and \(\textbf{P}(X, S)\).
    That is, for any set \(S\), there exists a set \(P\) such that \(X \in P\) if and only if \(X \subseteq S\).
  \end{solution}


  \section{Elementary Operations on Sets}

  % Problem 4.1
  \begin{problem}
    Prove all the displayed formulas in this section and visualize them using Venn diagrams.
  \end{problem}

  \begin{solution}
    I'll omit the Venn diagrams but I may return and add them later.
    \begin{enumerate}[label=({\alph*})]
      \item Commutativity:
      \begin{gather*}
        A \cap B = B \cap A \\
        A \cup B = B \cup A
      \end{gather*}

      \begin{proof}\
        Let \(x \in A \cap B\). Then \(x \in A\) and \(x \in B\). That is, \(x \in B\) and \(x \in A\) so \(x \in B \cap A\).
        The reverse argument is identical so \(A \cap B = B \cap A\). A similar argument proves that \(A \cup B = B \cup A\).
      \end{proof}

      \item Associativity:
      \begin{gather*}
        (A \cap B) \cap C = A \cap (B \cap C) \\
        (A \cup B) \cup C = A \cup (B \cup C)
      \end{gather*}

      \begin{proof}
        Let \(x \in (A \cap B) \cap C\). Then \(x \in A \cap B\) and \(x \in C\). That is, \(x \in A\), \(x \in B\), and \(x \in C\). So \(x \in B \cap C\). Therefore, \(x \in A \cap (B \cap C)\).
        The reverse argument is identical and the proof is similar for set union.
      \end{proof}

      \item Distributivity:
      \begin{gather*}
        A \cap (B \cup C) = (A \cap B) \cup (A \cap C) \\
        A \cup (B \cap C) = (A \cup B) \cap (A \cup C)
      \end{gather*}

      \begin{proof}
        Let \(x \in A \cap (B \cup C)\). Then \(x \in A\) and \(x \in B \cup C\). That is, \(x \in B\) or \(x \in C\).
        Therefore, \(x \in A\) and \(x \in B\) or \(x \in A\) and \(x \in C\). Thus, \(x \in (A \cap B) \cup (A \cap C)\).

        On the other hand, let \(x \in (A \cap B) \cup (A \cap C)\). Then \(x \in (A \cap B)\) or \(x \in (A \cap C)\).
        In the first case, \(x \in A\) and \(x \in B\), so \(x \in A\) and \(x \in (B \cup C)\). In the second case, \(x \in A\) and \(x \in C\), so \(x \in A\) and \(x \in (B \cup C)\).
        In either case, it follows that \(x \in A \cap (B \cup C)\). A highly similar argument proves the second statement.
      \end{proof}

      \item De Morgan's Laws:
      \begin{gather*}
        C - (A \cap B) = (C - A) \cup (C - B) \\
        C - (A \cup B) = (C - A) \cap (C - B)
      \end{gather*}

      \begin{proof}\renewcommand{\qedsymbol}{}
        Let \(x \in C - (A \cap B)\). Then \(x \in C\) and \(x \notin A \cap B\).
        That is, \(x \in C\), but \(x \notin A\) or \(x \notin B\). If \(x \notin A\), then \(x \in C - A\) so \(x \in (C - A) \cup (C - B)\).
        Similarly, if \(x \notin B\), then \(x \in C - B\) so \(x \in (C - A) \cup (C - B)\).

        For the reverse direction, let \(x \in (C - A) \cup (C - B)\). Then \(x \in C - A\) or \(x \in C - B\).
        In the first case, we have \(x \in C\) and \(x \notin A\), so \(x \notin (A \cap B)\). Thus, \(x \in C - (A \cap B)\).
        In the second case, we have \(x \in C\) and \(x \notin B\), so \(x \notin (A \cap B)\). Again, we have \(x \in C - (A \cap B)\). A similar argument proves the second statement.
      \end{proof}
    \end{enumerate}
  \end{solution}

  % Problem 4.2
  \begin{problem}
    Prove the following statements.
  \end{problem}

  \begin{solution}
    There are 7 :(
    \begin{enumerate}[label=({\alph*})]
      \item \(A \subseteq B\) if and only if \(A \cap B = A\) if and only if \(A \cup B = B\) if and only if \(A - B = \emptyset\)

      \begin{proof}
        Suppose \(A \subseteq B\). If \(x \in A \cap B\), then \(x \in A\) and \(x \in B\). In particular, \(x \in A\) so \(A \cap B \subseteq A\).
        If \(x \in A\), then \(x \in B\) since \(A \subseteq B\). Thus, \(x \in A \cap B\) so \(A \subseteq A \cap B\). This implies \(A \cap B = A\).

        Now suppose \(A \cap B = A\). If \(x \in B\), then \(x \in A \cup B\) by definition. Thus, \(B \subseteq A \cup B\).
        If \(x \in A \cup B\) then \(x \in A\) or \(x \in B\). If \(x \in A\), then \(x \in A \cap B\) so \(x \in B\).
        In either case, \(x \in B\) so \(A \cup B \subseteq B\). This implies \(A \cup B = B\).

        Assume that \(A \cup B = B\). Suppose \(A - B\) is nonempty and let \(x \in A - B\). Then \(x \in A\) and \(x \notin B\).
        If \(x \in A\) then \(x \in A \cup B\). However, if \(x \notin B\) and \(A \cup B = B\), we have \(x \notin A \cup B\).
        A contradiction arises so we must have \(A - B = \emptyset\).

        Finally, assume \(A - B = \emptyset\). Let \(x \in A\). Then either \(x \in B\) or \(x \notin B\).
        If \(x \notin B\), then we have \(x \in A\) and \(x \notin B\), so \(x \in A - B\), a contradiction.
        Thus, \(x \in B\) and \(A \subseteq B\).
      \end{proof}

      \item \(A \subseteq B \cap C\) if and only if \(A \subseteq B\) and \(A \subseteq C\).

      \begin{proof}
        First suppose \(A \subseteq B \cap C\). Let \(x \in A\). Then \(x \in B \cap C\), so \(x \in B\) and \(x \in C\). Thus, \(A \subseteq B\) and \(A \subseteq C\).

        For the reverse direction, suppose \(A \subseteq B\) and \(A \subseteq C\). Let \(x \in A\). Then \(x \in B\) and \(x \in C\), so \(x \in B \cap C\). Hence, \(A \subseteq B \cap C\).
      \end{proof}

      \item \(B \cup C \subseteq A\) if and only if \(B \subseteq A\) and \(C \subseteq A\).

      \begin{proof}
        Suppse \(B \cup C \subseteq A\). If \(x \in B\) then \(x \in B \cup C\), so \(x \in A\). Thus, \(B \subseteq A\). Similarly, \(C \subseteq C\).

        Now suppose \(B \subseteq A\) and \(C \subseteq A\). Let \(x \in B \cup C\). Then \(x \in B\) or \(x \in C\).
        In either case, \(x \in A\), so \(B \cup C \subseteq A\).
      \end{proof}

      \item \(A - B = (A \cup B) - B = A - (A \cap B)\)

      \begin{proof}
        Let \(x \in A - B\). That is, \(x \in A\) and \(x \notin B\). Then \(x \in A \cup B\) and \(x \notin B\) so \(x \in (A \cup B) - B\). Thus, \(A - B \subseteq (A \cup B) - B\).
        Now let \(x \in (A \cup B) - B\) so that \(x \in (A \cup B)\) and \(x \notin B\). It follows that \(x \in A\) and \(x \notin B\) so \(x \in A - B\) and \((A \cup B) - B \subseteq A - B\).
        This implies that \(A - B = (A \cup B) - B\).

        Suppose \(x \in A - B\). It follows that \(x \in A\) and \(x \notin B\). Then \(x \notin A \cap B\) so \(x \in A - (A \cap B)\). Hence, \(A - B \subseteq A - (A \cap B)\).
        For the reverse, suppose \(x \in A - (A \cap B)\). That is, \(x \in A\) and \(x \notin A \cap B\). Then \(x \notin B\) so \(x \in A - B\), showing that \(A - (A \cap B) \subseteq A - B\).
        Together, it is implied that \(A - B = A - (A \cap B)\).
      \end{proof}

      \item \(A \cap B = A - (A - B)\).

      \begin{proof}
        Let \(x \in A \cap B\). Then \(x \in A\) and \(x \in B\). It follows that \(x \notin A - B\). Thus, \(x \in A - (A - B)\) so \(A \cap B \subseteq A - (A - B)\).

        Now let \(x \in A - (A - B)\). That is, \(x \in A\) and \(x \notin A - B\). The latter implies that \(x \in B\) (since otherwise it would be in the difference).
        Therefore, \(x \in A \cap B\) so \(A - (A - B) \subseteq A \cap B\) and the two sets are equal.
      \end{proof}

      \item \(A - (B - C) = (A - B) \cup (A \cap C)\).

      \begin{proof}
        Let \(x \in A - (B - C)\). That is, \(x \in A\) and \(x \notin B - C\).
        Suppose that \(x \notin (A - B) \cup (A \cap C)\). Then it must be the case that \(x \in B\) and \(x \notin C\). If this were the case, \(x \in B - C\).
        By contraposition, if \(x \notin B - C\) it follows that \(x \in (A - B) \cup (A \cap C)\). Therefore, \(A - (B - C) \subseteq (A - B) \cup (A \cap C)\).

        Now let \(x \in (A - B) \cup (A \cap C)\) so \(x \in A - B\) or \(x \in A \cap C\).
        In the first case, \(x \in A\) and \(x \notin B\). If \(x \notin B\) then \(x \notin B - C\), so \(x \in A - (B - C)\).
        In the second case, \(x \in A\) and \(x \in C\). If \(x \in C\) then \(x \notin B - C\), so \(x \in A - (B - C)\).
        Thus, either case shows that \((A - B) \cup (A \cap C) \subseteq A - (B - C)\). Together, these statements imply \(A - (B - C) = (A - B) \cup (A \cap C)\).
      \end{proof}

      \item \(A = B\) if and only if \(A \triangle B = \emptyset\).

      \begin{proof}\renewcommand{\qedsymbol}{}
        Assume \(A = B\). Recall that \(A \triangle B = (A - B) \cup (B - A)\).
        \(A = B\) implies \(A \subseteq B\) and \(B \subseteq A\), and in (a) we proved that \(A \subseteq B\) if and only if \(A - B = \emptyset\).
        It follows that \(A - B = \emptyset\) and \(B - A = \emptyset\), so \(A \triangle B = \emptyset\).

        Now assume \(A \triangle B = \emptyset\). That is, \(A - B = \emptyset\) and \(B - A = \emptyset\). Again by (a), this is the case if and only if \(A \subseteq B\) and \(B \subseteq A\), so we have \(A = B\).
      \end{proof}
    \end{enumerate}
  \end{solution}

  % Problem 4.3
  \begin{problem}
    For each of the following (false) statements draw a Venn diagram in which it fails:
  \end{problem}

  \begin{solution}
    I'll add this when I'm more familiar with making Venn diagrams in \LaTeX (AKA I'm procrastinating)
  \end{solution}

  % Problem 4.4
  \begin{problem}
    Let \(A\) be a set; show that a ``complement" of \(A\) does not exist. (The ``complement" of \(A\) is the set of all \(x \notin A\).)
  \end{problem}

  \begin{solution}
    Suppose such a set exists, say \(B\). Then consider \(A \cup B\) to get the universal set, which does not exist. Thus \(B\) does not exist.
  \end{solution}

  % Problem 4.5
  \begin{problem}
    Let \(S \neq \emptyset\) and \(A\) be sets.
    \begin{enumerate}[label=({\alph*})]
      \item Set \(T_{1} = \{ Y \in \mathcal{P}(A) \mid Y = A \cap X \text{ for some } X \in S\}\), and prove \(A \cap \bigcup S = \bigcup T_{1}\) (generalized distributive law).
      \item Set \(T_{2} = \{Y \in \mathcal{P}(A) \mid Y = A - X \text{ for some } X \in S\}\), and prove
      \begin{gather*}
        A - \bigcup S = \bigcap T_{2} \\
        A - \bigcap S = \bigcup T_{2}
      \end{gather*}
      (generalized De Morgan laws).
    \end{enumerate}
  \end{problem}

  \begin{solution}
    Recall that \(S \neq \emptyset\).
    \begin{enumerate}[label=({\alph*})]
      \item \(x \in A \cap \bigcup S\) if and only if \(x \in A\) and \(\exists X \in S\) such that \(x \in X\).
      That is, \(x \in A \cap \bigcup S\) if and only if \(x \in T_{1}\).

      \item Let \(x \in A - \bigcup S\). That is, \(x \in A\) and \(x \notin \bigcup S\). Thus, for all \(X \in S\), we have \(x \notin X\).
      Therefore, for all \(X \in S\), we have \(x \in A - X\). Hence, \(x \in A - \bigcup S\) if and only if \(x \in \bigcap T_{2}\).

      Similarly, let \(x \in A - \bigcap S\). Then \(x \in A\) and \(\exists X \in S\) such that \(x \notin X\).
      That is, \(\exists Y \in \mathcal{P}(A)\) such that \(x \in Y\). Thus, \(x \in A - \bigcap S\) if and only if \(x \in \bigcup T_{2}\).
    \end{enumerate}
  \end{solution}

  % Problem 4.6
  \begin{problem}
    Prove that \(\bigcap S\) exists for all \(S \neq \emptyset\). Where is the assumption \(S \neq \emptyset\) used in the proof?
  \end{problem}

  \begin{solution}
    If \(S \neq \emptyset\), then \(\exists A \in S\). Consider the property \(\textbf{P}(x)\): ``\(x \in X\) for all \(X \in S\)."
    Then \(\bigcap S = \{x \in A \mid \textbf{P}(x)\}\) exists by the Axiom Schema of Comprehension.

    If \(S = \emptyset\), then \(\bigcap S\) is a universal set. Indeed, if it were not then there would exist a set \(A \in \emptyset\) such that \(x \notin A\), but no such set exists.
  \end{solution}

\end{document}
